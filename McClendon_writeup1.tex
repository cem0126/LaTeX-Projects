\documentclass[12pt,std, fleqn]{article}

\usepackage{amsmath,amsthm,amsfonts,amssymb,latexsym, hyperref, changepage}

\newtheorem{theorem} {Theorem}

\newtheorem{definition} {Definition}


\begin{document}
\
\begin{center}
\textbf{Caroline McClendon} \\
\vspace{.6 cm}
\textbf{Latex Write Up 1} \\
\textbf{June 12 2020}
\end{center}
\vspace{.2 cm}


\begin{theorem}[2.1 b]
Suppose that each of $ a, b$ and $c$ is an integer
a. $$a + (b + c) = (c + a) + b$$
\end{theorem} 

\begin{proof}
\begin{align*}
a + (b + c)&=(c + a) + b & \\
       &=a + (c + b) & \text{commutivity of addition}\\
       &=(a + c) + b & \text{associativity of addition}\\
       &=(c + a) + b & \text{commutivity of addition}
\end{align*}
\end{proof}
     
\begin{theorem}[2.1 e] Suppose that each of $a, b$ and $c$ is an integer.
$$(a+b)^2 = a^2 + 2ab + b^2.$$
\end{theorem}

\begin{proof} 
\begin{align*}
(a+b)^2 &= a^2 + 2ab + b^2 \\
        &= (a+b)(a+b)  & \text{definition of $x^2$}\\
        &= (a+b) \cdot a + (a+b) \cdot b & \text{distributive axiom} \\
        &= a(a+b) + b(a+b) & \text{commutivity of multiplication} \\
        &= a\cdot a + a\cdot a + b \cdot a + b \cdot b & \text{distributive}\\
        &= a^2 + a\cdot a + b \cdot a + b^2 & \text{defintion of $x^2$} \\
        &= a^2 + a\cdot a + a \cdot b + b^2 & \text{commutivity of multiplication}\\
        &= a^2 + 1ab + 1ab + b^2 &\text{multiplicative identity}\\
        &= a^2 + (1+1)ab + b^2 & \text{distributive axiom}\\
        &= a^2 + 2ab + b^2 
\end{align*}
\end{proof}
\pagebreak

\begin{theorem}[2.1 f] Suppose that each of $a, b$ and $c$ is an integer.
$$0 \cdot a = 0.$$
\end{theorem}

\begin{proof} 
\begin{align*}
0 \cdot a &= 0 \\
        &= (0 + 0) \cdot a  & \text{additive identity}\\
        &= 0\cdot a + 0 \cdot a & \text{distributive axiom} \\
0 \cdot a + (-0\cdot a) &= 0\cdot a + 0 \cdot a + (-0\cdot a) & \text{closure, additive inverse} \\
     0  &= 0\cdot a + (0 \cdot a + (-0\cdot a)) & \text{associativity of addition}\\
     0  &= 0\cdot a + 0 & \text{additive inverse} \\
     0  &= 0 \cdot a & \text{additive identity}\\
 0 \cdot a  &= 0 &\text{logic}
\end{align*}
\end{proof}

\begin{theorem}[2.1 g] Suppose that each of $a, b$ and $c$ is an integer.
$$ \mbox{If}\  a+b = 0\  \mbox{and}\  a+c=0, \mbox{then}\  b=c $$.
\end{theorem}
\vspace{-1 cm}

\begin{proof} 
\begin{align*}
a + b &= a + c & \text{property of equation sign, logic}\\
-a + a + b &= -a + a + c  & \text{closure, logic}\\
(-a + a) + b &= (-a + a) + b & \text{associativity} \\
0 + b &= 0 + c  & \text{additive inverse} \\
   b  &= c & \text{additive identity}\\
\end{align*}
\end{proof}
\pagebreak

\begin{theorem}[2.1 k] Suppose that each of $a, b$ and $c$ is an integer.
$$ -0 = 0 $$.
\end{theorem}
\vspace{-1 cm}

\begin{proof} 
\begin{align*}
-0 &= 0 \\
   &= (0 + -0)  & \text{additive identity}\\
&= 0 & \text{additive inverse} \\
\end{align*}
\end{proof}


\begin{theorem}[3.2] If n $\in \mathbb{N} $ then,
$$ \sum_{i=1}^{n} i^2 = \dfrac{n(n+1)(2n+1)}{6} $$.
\end{theorem}
\vspace{-1 cm}

\begin{proof} 
\text{Prove}\  $P_{1}$\  \text{is true: (let n = 1)}
\begin{align*}
 \text{left side} &=  1^2 = 1\\
 \text{right side} &= \dfrac{1 \cdot 2 \cdot 3}{6} = 1 \\
\mbox{Prove}\ P_{N} \to P_{N} + 1 :\\
\sum_{i=1}^{n} i^2 &= \dfrac{n(n+1)(2n+1)}{6} \\
\sum_{i=1}^{n+1} i^2 &= \dfrac{n(n+1)(2n+1)}{6} + (n+1)^2 \quad \text{induction hypothesis} \\
&= \dfrac{n+1(n(2n+1) + 6(n+1))}{6} \\
&= \dfrac{(n+1)((2n+3)(n+2))}{6} \\
&= \frac{1}{6}(n+1)(n+2)(2(n+1)+1) 
\end{align*}
\end{proof}



\end{document}

